\section{Topography layer description file (topology.tpl)}

Each line of the topography layer description file (topology.tpl) contains
a comma separated list (CSV) of information needed for rendering of an
individual topography layer. Lines starting with '*' are ignored.

XCSoar v6.6 and earlier will display at most 20 topography layers.
XCSoar v6.7 and later will display at most 30 topography layers.
\begin{table}[ht]
\centering
\sffamily
\begin{tabular}{@{}lll@{}}
\toprule
\addlinespace
Column name&Data type&Valid range\\
%\midrule
\cmidrule(r){1-1}\cmidrule(lr){2-2}\cmidrule(l){3-3}
filename&string&\\
range&double (nm)&-\\
icon&string&\\
label index&int&0-1\\
color (red component)&int&0-255\\
color (green component)&int&0-255\\
color (blue component)&int&0-255\\
pen width&int&0-31\\
label range (nm)&double&-\\
important label range (nm)&double&-\\
alpha&int&0-255\\
%\addlinespace
\bottomrule
\end{tabular}
\caption{Topography file format}
\label{tab:topography-file-format}
\end{table}

\begin{description}
\item[filename] The filename of the Topography layer within the container file.
\item[icon] XCSoar v6.5 and earlier, Only the value 219 is recognised, for town icons.
From XCSoar v6.6, the name of the icon to display. Optional. See below for a list of available names.
\item[range] Zoom level threshold. Layer elements will not be drawn unless zoomed in closer than this threshold.
\item[pen width] Lines contained within this layer are drawn with pen width.
\item[label range] Label display zoom level threshold. Labels contained in the layer file
will not be rendered unless zoomed in closer than this threshold.
\item[important label range] A zoom level threshold. Labels contained in the layer file will be
rendered in standard style when the display zoom level is greater than this threshold.
\item[alpha] The alpha component controls transparency of polygons...
0 means polygons are completely transparent, 255 means they are completely opaque.
Only used by XCSoar v6.7 and later.

\texttt{Versions of XCSoar running on Windows and WinCE ignore any item where transparency is specified.}

\end{description}
\subsection{Point Features}
Prior to XCSoar v6.6, this could contain the value 219 to display an icon for a town
From XCSoar v6.6, a user can put an optional string into the icon column in topology.tpl in the .XCM file (e.g.)
\begin{itemize}
\item SpotHeight,5,\texttt{mountain\_top},1,64,64,64,1,5,
\item Mast,10,\texttt{obstacle},,,,,1,10,
\end{itemize}
This can be used for Shapefiles containing point features or polygons or linestrings, but is probably only useful for point features.

The icon of the corresponding image and optional label will be displayed. In the first example, 
the "mountain\_top" icon and a label will be displayed for each point in the SpotHeight shapefile. My
SpotHeight Shapefile has been generated with the point elevation in feet as the label value). For the second example, only "obstacle" icons 
(no labels) will be displayed for points in the Mast Shapefile..

Icon names are detected in TopographyStore.cpp. Names must be given in lowercase. If the icon name given is unknown, or no icon name is given, then icons are not displayed for that Shapefile.


Names correspond to images which have been linked into XCSoar, although
it is envisaged that in future these will be names of icon files. Available icon names are:
\begin{itemize}
\item mountain\_top \includegraphics[angle=0,width=8pt,keepaspectratio='true']{icons/map_mountain_top.pdf}
\item bridge \includegraphics[angle=0,width=8pt,keepaspectratio='true']{icons/map_bridge.pdf}
\item tunnel \includegraphics[angle=0,width=8pt,keepaspectratio='true']{icons/map_tunnel.pdf}
\item tower \includegraphics[angle=0,width=8pt,keepaspectratio='true']{icons/map_tower.pdf}
\item power\_plant \includegraphics[angle=0,width=8pt,keepaspectratio='true']{icons/map_power_plant.pdf}
\item obstacle \includegraphics[angle=0,width=8pt,keepaspectratio='true']{icons/map_obstacle.pdf}
\item mountain\_pass \includegraphics[angle=0,width=8pt,keepaspectratio='true']{icons/map_pass.pdf}
\item weather\_station \includegraphics[angle=0,width=8pt,keepaspectratio='true']{icons/map_weather_station.pdf}
\item thermal\_hotspot
\item town
\item mark \includegraphics[angle=0,width=8pt,keepaspectratio='true']{icons/map_flag.pdf}
\item turnpoint \includegraphics[angle=0,width=8pt,keepaspectratio='true']{icons/map_turnpoint.pdf}
\item small
\item cruise \includegraphics[angle=0,width=8pt,keepaspectratio='true']{icons/mode_cruise.pdf}
\item terrainwarning
\item logger
\item loggeroff
\item target
\item teammate\_pos
\item airspacei
\item traffic\_safe
\item traffic\_warning
\item traffic\_alarm
\item taskturnpoint
\item marginal \includegraphics[angle=0,width=8pt,keepaspectratio='true']{icons/winpilot_marginal.pdf}
\item landable \includegraphics[angle=0,width=8pt,keepaspectratio='true']{icons/winpilot_landable.pdf}
\item reachable \includegraphics[angle=0,width=8pt,keepaspectratio='true']{icons/winpilot_reachable.pdf}
\item airport\_reachable \includegraphics[angle=0,width=8pt,keepaspectratio='true']{icons/alt_reachable_airport.pdf}
\item airport\_unreachable \includegraphics[angle=0,width=8pt,keepaspectratio='true']{icons/alt_landable_airport.pdf}
\item airport\_marginal \includegraphics[angle=0,width=8pt,keepaspectratio='true']{icons/alt_marginal_airport.pdf}
\item airport\_unreachable2 \includegraphics[angle=0,width=8pt,keepaspectratio='true']{icons/alt2_landable_airport.pdf}
\item airport\_marginal2 \includegraphics[angle=0,width=8pt,keepaspectratio='true']{icons/alt2_marginal_airport.pdf}
\item outfield\_unreachable2 \includegraphics[angle=0,width=8pt,keepaspectratio='true']{icons/alt2_landable_field.pdf}
\item outfield\_marginal2 \includegraphics[angle=0,width=8pt,keepaspectratio='true']{icons/alt2_marginal_field.pdf}
\item outfield\_reachable \includegraphics[angle=0,width=8pt,keepaspectratio='true']{icons/alt_reachable_field.pdf}
\item outfield\_unreachable \includegraphics[angle=0,width=8pt,keepaspectratio='true']{icons/alt_landable_field.pdf}
\item outfield\_marginal \includegraphics[angle=0,width=8pt,keepaspectratio='true']{icons/alt_marginal_field.pdf}
\end{itemize}
\subsection{Adding new Icons}
At the moment, adding new icons requires a rebuild of the XCSoar application.It is envisaged that, in future, 
this process won't be required... users will include icon files in their .XCM map container files, and refer to them 
by name. However, that has not yet been implemented. 

To add your own images to the list of icons:
\begin{enumerate}
\item Create a .svg file for the icon (e.g. mast.svg) and copy into \texttt{xcsoar/Data/icons}. For Android, the name must be lowercase.
\item Insert two (for normal and high-res) lines into \texttt{xcsoar/Data/XCSoar.rc},  (e.g.)
\begin{verbatim}
BITMAP\_ICON(IDB\_MAST, "mast")
BITMAP\_ICON(IDB\_MAST\_HD, "mast\_160")
\end{verbatim}
\item Insert two lines into \texttt{xcsoar/src/resource.h} (e.g.)
\begin{verbatim}
\#define IDB\_MAST                        500
\#define IDB\_MAST\_HD                    5500
\end{verbatim}
\item Add a corresponding line into the \texttt{icon\_list} table in \texttt{xcsoar/src/Topography/TopographyStore.cpp}
\begin{verbatim}
  {"mast", IDB\_MAST},
\end{verbatim}
\item Make XCSoar
\end{enumerate} 
After this, a line can be added in \texttt{topology.tpl} to connect the icon to the Shapefile using the icon name. (e.g.)
\begin{verbatim}
Mast,10,mast,,,,,1,10,
\end{verbatim}

Note that unless these changes are merged into the main XCSoar repository, then only your specific build of XCSoar will be able to 
display your icon image.
