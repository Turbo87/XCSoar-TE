\chapter{Data Files}\label{cha:data-files}

Data files used by XCSoar fall into two categories:
\begin{description}
\item[Flight data files]  These files contain data relating to
the aircraft type, airspace and maps, waypoints etc.  These are the
files that are most likely to be modified or set by normal users.
\item[Program data files]  These files contain data relating to
the `look and feel' of the program,
button assignments, input events.
\end{description}
This chapter focuses on flight data files; see the {\em XCSoar
Advanced Configuration Guide} for details on program data files.

\section{File management}

File names must correspond to the name extensions specified below.  It
is good practice to make sure that the file names are recognisable so
that when making configuration changes there is less risk of confusion
between different files and different file types.

Regarding older Pocket PC devices it is a good idea to have data files located in
nonvolatile memory, the use of SD cards and other removable media in
PDAs can cause performance issues; for smaller files, and files that
are only accessed at start-up (waypoints, airspace, glide polars,
configuration files), this is acceptable.  However, terrain and
topography files are accessed continuously while XCSoar is running, so
these should be located in faster storage memory.
For newer Windows Mobile or Android devices this is not an issue any 
more. The access to modern memory cards usually meets the required performance.   

Many PDAs provide a 'file store' which is nonvolatile; the same
arguments above apply regarding their use and performance.

All data files should be copied into the directory: 
\begin{verbatim}
My Documents/XCSoarData
\end{verbatim}

On PDAs data can also be stored on the operating system file
store, on Compact Flash cards or SD cards under the directory
\verb|XCSoarData|.

For example:
\begin{verbatim}
SD Card/XCSoarData
IPAQ File Store/XCSoarData
\end{verbatim}

If unsure, just start the newly installed XCSoar and it creates the \verb|XCSoarData|
directory at the right place.

\section{Map Database}\label{sec:map}

A map database (extension \verb|.xcm|) contains terrain,
topography and optional contents like waypoints and airspaces.

Terrain is a raster digital elevation model represented as an array of
elevations in meters on a latitude/longitude grid.  The internal file
format is GeoJPEG2000.

The topography is vector data such as roads, railway lines, large
built-up areas (cities), miscellaneous populated areas (towns and
villages), lakes and rivers.  The topography is stored in ESRI Shape
files which are generated from OpenStreetMap.

Map files can be downloaded from the XCSoar web site:

\url{http://www.xcsoar.org/download/maps/}

To generate a custom map database with different settings and bounds, you
may use the map generator:

\url{http://mapgen.xcsoar.org/}

As far as waypoints or airspaces are included in the map database CXSoar 
defaults to them. An e.g. separately configured waypoint file will replace all the 
waypoints given by the map database.

\section{Waypoints}

XCSoar understands the following waypoint file formats:

\begin{itemize}
\item WinPilot/Cambridge (\verb|.dat|)
\item SeeYou (\verb|.cup|)
\item Zander (\verb|.wpz|)
\item OziExplorer (\verb|.wpt|)
\item GPSDump/FS, GEO and UTM (\verb|.wpt|)
\end{itemize}

Files are available from the Soaring Turn-points section of the
Soaring Server\footnote{Mirrors to this website exist, google search
for ``worldwide soaring turnpoint exchange'' if the main server is
inaccessible.}: \url{http://soaringweb.org/TP}

Several commercial and freely distributable programs exist for
converting between different waypoint formats.

If the elevation of any waypoints is set to zero in the waypoint file,
then XCSoar estimates the waypoint elevation from the terrain database
if available.

\section{Airspace}

XCSoar supports airspace files (extension \verb|.txt|) using a sub set
of the widely distributed OpenAir format, as well as the Tim Newport-Pearce file
format (extension \verb|.sua|). Files are available from the
Special Use Airspace section of the soaring web site:

\url{http://soaringweb.org/Airspace}

The following is the list of supported airspace types: Class A-G, Prohibited, 
Danger Areas, Restricted, Task Area, CTR, No Gliders, Wave, Transponder 
Mandatory, and Other.  All other airspace types will be drawn as type ``Other''.
In addition to the OpenAir standard the AR command is recognized as the 
airspace radio frequency.

\section{Airfield details}\label{sec:airfield-details}

The airfield details file (extension \verb|.txt|) is a simple text
format file containing entries for each airfield, marked in square
brackets, followed by the text to be displayed on the
Waypoint Details Dialogue for that particular waypoint.  The text should
have a narrow margin because the waypoint details dialogue cannot
currently handle word wrapping.

The text may also specify images for airfields or waypoints.  To
show an image directly in XCSoar use \verb|image=| followed by the file
name (this is currently not supported on PC/Windows).  Be sure
to avoid any additional whitespaces around the equal sign or in front of the
keyword. Which files are supported depends on your operating system and the
applications that are installed.  Android supports JPEG files and other
file types, others mostly BMP images.

The names of airfields used in the file must correspond exactly to the
names in the waypoints file, with the exception that converting to
uppercase is allowed.

The XCSoar website provides airfield details files for several
countries and includes tools to convert from various Enroute
Supplement sources to this file format.

Users are free to edit these files to add their own notes for
airfields that may not otherwise be included in the Enroute Supplement
sources.

An example (extract from the Australian airfields file):
\begin{verbatim}
[BENALLA]
RUNWAYS:
  08 (RL1,7) 17 (RL53) 26
  (R) 35 (R)

COMMUNICATIONS:
  CTAF - 122.5 REMARKS: Nstd
  10 NM rad to 5000'

REMARKS:
  CAUTION - Animal haz. Rwy
  08L-26R and 17L-35R for
  glider ops and tailskidacft
  only, SR-SS. TFC PAT - Rgt
  circuits Rwy 08R-26L. NS
  ABTMT - Rwy 17R-35L fly wide

ICAO: YBLA

image=Benalla_sat.bmp

[GROOTE EYLANDT]
Blah blah blah blah
...
\end{verbatim}

\section{Glide polar} \label{sec:glide-polar}

Many polars of common gliders are built into XCSoar.  If your glider
model is not listed, you can use a polar file in the WinPilot polar
format (extension \verb|.plr|).

The WinPilot and XCSoar websites provide several glide polar files.
Files for other gliders may be created upon request to the XCSoar
team.

The format of the file is simple.  Lines beginning with \verb|*| are
ignored and so may be used to document how the polar was calculated or
if there are restrictions on its use.  Other than comments, the file
must contain a single row of numbers separated with commas:
\begin{itemize}
\item Mass dry gross weight in kg: this is the weight of the glider plus
  a 'standard' pilot without ballast.
\item Max water ballast in liters (kg).
\item Speed in km/h for first measurement point, (usually minimum sink speed).
\item Sink rate in m/s for first measurement point.
\item Speed in km/h for second measurement point, (usually best glide speed).
\item Sink rate in m/s for second measurement point.
\item Speed in km/h for third measurement point, (usually max manoeuvring speed).
\item Sink rate in m/s for third measurement point.
\end{itemize}
The following is an extension to the existing polare file format and thus optional. 
\begin{itemize}
\item The wing area in m$^2$ to allow the wing load computation (could be zero if unknown). 
\item The max. manoeuvering speed in km/h to enable simple checks for the cruise speed command. 
\end{itemize}

An example, for the LS-3 glider, is given below:
\begin{verbatim}
*LS-3	WinPilot POLAR file: MassDryGross[kg], 
*  MaxWaterBallast[liters], Speed1[km/h], Sink1[m/s], 
*  Speed2, Sink2, Speed3, Sink3  	
373,	121,	74.1,	-0.65,	102.0,	-0.67,	167.0,	-1.85
\end{verbatim}

\tip Don't be too optimistic when entering your polar data. It is all too
easy to set your LD too high and you will rapidly see yourself
undershooting on final glide.

\section{Profiles}

Profile files (extension \verb|.prf|) can be used to store
configuration settings used by XCSoar.  The format is a simple text
file containing \verb|<label>=<value>| pairs.  Certain values are text
strings delimited by double quotes, for example:
\begin{verbatim}
PilotName="Baron Richtoffen"
\end{verbatim}
All other values are numeric, including ones that represent boolean
values (true$=1$, false$=0$), for example:
\begin{verbatim}
StartDistance=1000
\end{verbatim}

All values that have physical dimensions are expressed in SI units
(meters, meters/second, seconds etc).

When a profile file is saved, it contains all configuration settings.
Profile files may be edited with a text editor to produce a smaller
set of configuration settings that can be given to other pilots to
load.  

When a profile file is loaded, only the settings present in that file
overwrite the configuration settings in XCSoar; all other settings are
unaffected.

The default profile file is generated automatically when configuration
settings are changed or when the program exits; this has the
file name \verb|default.prf|.

The easiest way to create a new profile is to copy a previous one,
such as the default profile.  Copy the file, give it a logical name,
and then when XCSoar starts next time the new profile can be selected
and customised through the configuration settings dialogs.


\section{Checklist}\label{sec:checklist-file}

The checklist file (\verb|xcsoar-checklist.txt|) uses a similar format to
the airfield details file.  Each page in the checklist is preceded by
the name of the list in square brackets.  Multiple pages can be
defined (up to 20).

An example (extract):
\begin{verbatim}
[Preflight]
Controls
Harness, secure objects
Airbrakes and flaps
Outside
Trim and ballast
Instruments
Canopy

[Derigging]
Remove tape from wings and tail
...
\end{verbatim}

\section{Tasks}

Task files (extension \verb|.tsk|) are stored in a XCSoar own XML format, 
whereas SeeYou tasks can also be loaded from file (extension \verb|.cup|).

\section{Flight logs} \label{sec:logfiles}

The software flight logger generates IGC files (extension \verb|.igc|)
according to the long naming convention described in the FAI 
document {\em Technical Specification for IGC-Approved GNSS Flight Recorders}.  

Log-files are stored in the `logs' subdirectory of the XCSoarData directory.
These files can be imported into other programs for analysis after flight.

\section{FLARM Identification}\label{sec:flarm-ident-file}

The FLARM identification file \verb|xcsoar-flarm.txt| defines a table
of aircraft registrations or pilot names against the ICAO IDs that are
optionally broadcast by FLARM equipped aircraft.  These names are
displayed on the map next to FLARM traffic symbols, for matching ICAO
IDs.

The format of this file is a list of entries, one for each aircraft,
of the form {\em icao id=name}, where {\em icao id} is the six-digit
hex value of the ICAO aircraft ID, and {\em name} is free text
(limited to 20 characters), describing the aircraft and/or pilot name.
Short names are preferred in order to reduce clutter on the map
display.

Example:
\begin{verbatim}
DD8F12=WUS
DA8B06=Chuck Yeager
\end{verbatim}

Currently this file is limited to a maximum of 200 entries.

Additionally the FlarmNet file \verb|data.fln| is supported. It contains all 
the FLARM identifications distributed by the FlarmNet community. The file can 
be downloaded from the web site:

\url{http://www.flarmnet.org}

The file must reside in the XCSoarData directory.


%%%%%%%%%%%%%% advanced stuff below..
\section{Input events}

The input event file (extension \verb|.xci|) is a plain text file
designed to control the input and events in your glide computer.

You do not require access to the source code or understanding of
programming to write your own input event files but you do require
some advanced understanding of XCSoar and of gliding.

Some reasons why you might like to use xci:
\begin{itemize}
\item Modify the layout of button labels
\item Support a new set or layout of buttons (organiser hardware buttons)
\item Support an external device such as a bluetooth keyboard or gamepad
\item Customise any button/key event
\item Do multiple events from one key or glide computer triggered process
\end{itemize}
For more information on editing or writing your own input event
file the {\em XCSoar Developer Manual} is a good starting point.


\section{Status}\label{sec:status-file}

Status files are text of the form {\em label=value}, arranged in
blocks of text where each block corresponds to an individual status
message.  These are delimited by double spaces.  Each block can
contain the following fields:
\begin{description}
\item[key]  This is the text of the status message.
\item[sound] Location of a WAV audio file to play when the status
  message appears.  This is optional.
\item[delay] Duration in milliseconds the status message is
  to be displayed.  This is optional.
\item[hide] A boolean (yes/no) that dictates whether the message
 is to be hidden (that is, not displayed). 
\end{description} 

Example:
\begin{verbatim}
key=Simulation\r\nNothing is real!
sound=\My Documents\XCSoarData\Start_Real.wav
delay=1500

key=Task started
delay=1500
hide=yes
\end{verbatim}
% 
