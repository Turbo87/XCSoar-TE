\chapter{Compiling XCSoar}\label{cha:compiling}

The \texttt{make} command is used to launch the XCSoar build process.
You can learn more about the build system internals in chapter
\ref{cha:build}.

Most of this chapter describes how to build XCSoar on Linux, with
examples for Debian/Ubuntu.  A cross-compiler is used to build
binaries for other operating systems (for example Android and
Windows).

\section{Getting the Source Code}

The XCSoar source code is managed with
\href{http://git-scm.com/}{git}.  It can be downloaded with the
following command:

\begin{verbatim*}
git clone git://git.xcsoar.org/xcsoar/master/xcsoar.git
\end{verbatim*}

To update your repository, type:

\begin{verbatim*}
git pull
\end{verbatim*}

For more information, please read to the git documentation.

\section{Requirements}

The following is needed for all targets:

\begin{itemize}
\item GNU make
\item GNU compiler collection (\texttt{gcc}), version 4.6.2 or later
  or clang/LLVM 3.2 (with "make CLANG=y")
\item GNU gettext
\item \href{http://librsvg.sourceforge.net/)}{rsvg}
\item \href{http://www.imagemagick.org/}{ImageMagick 6.4}
\item \href{http://xmlsoft.org/XSLT/xsltproc2.html}{xsltproc}
\item Perl and XML::Parser
\end{itemize}

The following command installs these on Debian:

\begin{verbatim*}
apt-get install make \
  librsvg2-bin xsltproc \
  imagemagick gettext
\end{verbatim*}

\section{Target-specific Build Instructions}

\subsection{Compiling for Linux/UNIX}

The following additional packages are needed to build for Linux and
similar operating systems:

\begin{itemize}
\item \href{http://www.zlib.net/}{zlib}
\item \href{http://curl.haxx.se/}{CURL}
\item \href{http://www.libsdl.org/}{SDL}
\item \href{http://www.libsdl.org/projects/SDL\_ttf/}{SDL\_ttf}
\item \href{http://www.libpng.org/}{libpng}
\item \href{http://libjpeg.sourceforge.net/}{libjpeg}
\item OpenGL (Mesa)
\item to run XCSoar, you need one of the following fonts (Debian
  package): DejaVu (\texttt{ttf-dejavu}), Droid (\texttt{ttf-droid}),
  Freefont (\texttt{ttf-freefont})
\end{itemize}

The following command installs these on Debian:

\begin{verbatim*}
apt-get install make g++ \
  zlib1g-dev \
  libsdl1.2-dev libfreetype6-dev \
  libpng-dev libjpeg-dev \
  libcurl4-openssl-dev \
  libxml-parser-perl \
  librsvg2-bin xsltproc \
  imagemagick gettext \
  ttf-dejavu
\end{verbatim*}

To compile, run:

\begin{verbatim*}
make
\end{verbatim*}

You may specify one of the following targets with \texttt{TARGET=x}:

\begin{tabular}{lp{8cm}}

\texttt{UNIX} & regular build (the default setting) \\

\texttt{UNIX32} & generate 32 bit binary \\

\texttt{UNIX64} & generate 64 bit binary \\

\texttt{OPT} & alias for UNIX with optimisation and no debugging \\

\end{tabular}

\subsection{Compiling for Android}

For Android, you need:

\begin{itemize}
\item \href{http://developer.android.com/sdk/}{Android SDK level 17}
\item \href{http://developer.android.com/sdk/ndk/}{Android NDK r8e}
\item \href{http://www.vorbis.com/}{Ogg Vorbis}
\item \href{http://ant.apache.org/}{Apache Ant}
\end{itemize}

The \texttt{Makefile} assumes that the Android SDK is installed in
\verb|~/opt/android-sdk-linux_x86| and the NDK is installed in
\verb|~/opt/android-ndk-r8e|.  You can use the options
\verb|ANDROID_SDK| and \verb|ANDROID_NDK| to override these paths.

Load/update the IOIO source code:

\begin{verbatim*}
git submodule init
git submodule update
\end{verbatim*}

To compile, run:

\begin{verbatim*}
make TARGET=ANDROID
\end{verbatim*}

Use one of the following targets:

\begin{tabular}{lp{8cm}}

\texttt{ANDROID} & for ARMv6 CPUs \\

\texttt{ANDROID7} & for ARMv7 CPUs \\

\texttt{ANDROID7NEON} & with
\href{http://www.arm.com/products/processors/technologies/neon.php}{NEON}
extension \\

\texttt{ANDROID86} & for x86 CPUs \\

\texttt{ANDROIDMIPS} & for MIPS CPUs \\

\texttt{ANDROIDFAT} & "fat" package for all supported CPUs \\

\end{tabular}

\subsection{Compiling for Windows}

To cross-compile to (desktop) Windows, you need the mingw-w64 version
of gcc:

 http://mingw-w64.sourceforge.net/

To compile, run one of the following:

\begin{verbatim*}
make TARGET=PC
\end{verbatim*}

Use one of the following targets:

\begin{tabular}{lp{8cm}}

\texttt{PC} & 32 bit Windows (i686) \\

\texttt{WIN64} & Windows x64 (amd64 / x86-64) \\

\texttt{WINE} & WineLib (experimental) \\

\texttt{CYGWIN} & Windows build with Cygwin (experimental) \\

\end{tabular}

\subsection{Compiling for Windows CE}

For PocketPC / Windows CE / Windows Mobile, you need mingw32ce:

\begin{itemize}
\item \href{http://max.kellermann.name/projects/cegcc/}{mingw32ce}
\end{itemize}

To compile, run:

\begin{verbatim*}
make TARGET=WM5X
\end{verbatim*}

Use one of the following targets:

\begin{tabular}{lp{8cm}}

\texttt{PPC2000} & PocketPC 2000 / Windows CE 3.0 \\

\texttt{PPC2003} & PocketPC 2003 / Windows CE 4.0 \\

\texttt{PPC2003X} & for XScale CPUs \\

\texttt{WM5} & Windows Mobile / Windows CE 5.0 \\

\texttt{WM5X} & for XScale CPUs \\

\texttt{ALTAIR} & for Triadis Altair \\

\end{tabular}

\subsection{Compiling for Mac OS X}

For Mac OS X, you need:

\begin{itemize}
\item GCC 4.6.2 or newer (http://hpc.sourceforge.net/, or homebrew, or Macports)
\item \href{http://www.zlib.net/}{zlib}
\item \href{http://curl.haxx.se/}{CURL}
\item \href{http://www.libsdl.org/}{SDL}
\item \href{http://www.libsdl.org/projects/SDL\_ttf/}{SDL\_ttf}
\item \href{http://www.libpng.org/}{libpng}
\item \href{http://libjpeg.sourceforge.net/}{libjpeg}
\item \href{http://icns.sourceforge.net/}{libicns}
\end{itemize}

\subsection{Compiling for the Raspberry Pi}

To compile, run:

\begin{verbatim*}
make TARGET=PI
\end{verbatim*}

This target is only used for cross-compiling on a (desktop) computer.
If you compile on the Raspberry Pi, the default target will
auto-detect the Pi.

\subsection{Compiling for Kobo E-book Readers}

Download the ARM cross-compiler and unpack in your home directory:

\begin{verbatim*}
wget http://max.kellermann.name/download/xcsoar/devel/x-tools-arm.tar.xz
tar xJfC x-tools-arm.tar.xz $HOME
export PATH=$HOME/x-tools/arm-unknown-linux-gnueabi/bin:$PATH
\end{verbatim*}

Download the Kobo libraries and unpack in \texttt{/opt}:

\begin{verbatim*}
wget http://max.kellermann.name/download/xcsoar/devel/kobo/kobo-libs-2013-07-17.tar.xz
tar xJfC kobo-libs-2013-07-17.tar.xz /opt
\end{verbatim*}

To compile, run:

\begin{verbatim*}
make TARGET=KOBO
\end{verbatim*}

\section{Options}

\subsection{Parallel Build}

Most contemporary computers have multiple CPU cores.  To take
advantage of these, use the \texttt{make -j} option:

\begin{verbatim*}
make -j12
\end{verbatim*}

This command launches 12 compiler processes at the same time.

Rule of thumb: choose a number that is slightly larger than the number
of CPU cores in your computer.  12 is a good choice for a computer
with 8 CPU cores.

\subsection{Optimised Build}

By default, debugging is enabled and compiler optimisations are
disabled.  The resulting binaries are very slow.  During development,
that is helpful, because it catches more bugs.

To produce optimised binaries, use the option \texttt{DEBUG}:

\begin{verbatim*}
make DEBUG=n
\end{verbatim*}

Be sure to clean the output directory before you change the
\texttt{DEBUG} setting, because debug and non-debug output files are
not compatible.

The convenience target \texttt{OPT} is a shortcut for:

\begin{verbatim*}
TARGET=UNIX DEBUG=n TARGET_OUTPUT_DIR=output/OPT
\end{verbatim*}

It allows building both debug and non-debug incrementally, because two
different output directories are used.
